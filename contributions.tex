\newpage
\section*{Contributions}

\paragraph*{陈 晨} 学号:2052717;贡献占比:12.5\%。


\paragraph*{戴仁杰} 学号:19516504;贡献占比:12.5\%。

在本项目中,我主要负责对traceLog模块前端进行设计,期间使用了数据访问对象模式,传输对象模式和MVP模式。

\paragraph*{姜文渊} 学号:1951510;贡献占比:12.5\%。

本项目中,我主要负责项目选题和文档排版的工作,兼后端的设计模式总体设计和进度的协调工作。

\paragraph*{李乐天} 学号:1950849;贡献占比:12.5\%。

在本项目中,我主要负责MVC模式的实现,以及空对象模式的实现,和项目整体的UI样式设计与实现,包括确定应用程序的整体外观、布局和样式。

\paragraph*{孟 宇} 学号:1951477;贡献占比:12.5\%。

​在本项目中,我主要负责了 \lstinline{CpuState} 中通用寄存器、栈指针、程序计数器的逻辑设计与实现。还包括 ansi 控制序列以及 CPU 运行日志打印部分的代码实现。其中主要涉及到的设计模式为享元模式、适配器模式、装饰器模式、原型模式以及策略模式。

\paragraph*{杨淳屹} 学号:1953824;贡献占比:12.5\%。

在本项目中,我主要负责工厂(合作完成),过滤器和迭代器模式的实现,负责部分后端代码的设计和实现,包括参与整体系统设计的讨论和分析

\paragraph*{杨孟臻} 学号:1953243;贡献占比:12.5\%。

本项目中,我主要负责\lstinline{CPU}和\lstinline{CpuState}对象及其代理的设计实现和封装、IR 寄存器及其子类、状态寄存器 \lstinline{FlagRegister}的设计实现和封装,以及项目后端设计模式的综合设计。实现的设计模式为:代理模式、桥接模式、外观模式、组合模式、访问者模式。

\paragraph*{杨 鑫} 学号:1950787;贡献占比:12.5\%。

在本项目中,我主要负责单例模式,工厂模式,抽象工厂模式,建造者模式,备忘录模式的实现,以及后端逻辑功能的实现。


\subsection{责任链模式(Chain of Responsibility)}

\subsubsection{责任链模式简介}

责任链模式是一种软件设计模式,它用于将一个请求的处理分派给多个对象,这些对象形成一条链,并依次处理该请求。

责任链模式的一个常见应用场景是实现一个处理请求的系统。例如,假设我们有一个类,它可以用于处理用户发送的消息。在这种情况下,我们可以使用责任链模式来处理消息,从而让消息能够在多个处理器之间传递,直到被处理为止。

责任链模式具有以下优点:
\begin{enumerate}
\item 责任链模式可以实现一种松耦合的设计,使得请求的发送者和处理者之间不需要直接引用来进行通信。
\item 责任链模式可以提高系统的灵活性,因为它允许在不更改对象之间的关系的情况下添加新的处理器。
\item 责任链模式可以使得系统更加模块化,从而更容易实现和维护。
\end{enumerate}

责任链模式也有一些缺点,包括:
\begin{enumerate}
\item 责任链模式可能会使得系统变得过于复杂,因为它需要维护请求发送者和处理者之间的关系。
\item 责任链模式可能会导致性能问题,因为它需要在多个对象之间进行额外的通信。
\item 责任链模式可能会导致循环依赖,特别是在请求发送者和处理者之间存在多对多的关系时。
\end{enumerate}

\subsubsection{责任链模式在项目中的应用}

在我们的项目中,责任链模式被用于处理异常,例如:
\begin{lstlisting}
    public Memory(int startAddress, int endAddress) throws MemoryRangeException {
        this(startAddress, endAddress, false);
    }
\end{lstlisting}

通过定义 \lstinline{catch (MemoryRangeException ex)},完成处理异常的责任的传递。
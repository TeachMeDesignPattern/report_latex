\section{项目总结}

\subsection{硬件模拟器开发的难点}

在本小组的 Slow6502 项目中,我们实现了理论课上讲到的所有 23 个 GoF 设计模式,以及 8 个不属于 GoF 的设计模式。这些设计模式使用并非生搬硬套,而是符合项目的实际情况的。相比较常规的业务软件,构建一个硬件平台的指令级模拟软件,主要面临以下几个难点:
\begin{enumerate}
    \item 硬件模拟器的行为十分复杂,且必须进行精准的复刻
    \item 硬件平台结构和组成十分复杂,且在各类文档中的抽象层次不一致
    \item 用户对于硬件模拟器的灵活性要求较高,需要能够模拟不同型号的处理机变种、内存、外设等
    \item 由于面向的用户大多为专业用户,故而硬件模拟器的源码需要有良好的可扩展潜力,以方便用户进行自行修改
    \item 硬件模拟器需要一定的执行效率
\end{enumerate}

目前最广泛使用的 QEMU 为了执行效率和模拟的精准性,而牺牲了源码的可维护性。考虑到目前的常见设备的计算能力通常是 MOS 6502 的数万倍,故而模拟器的性能并不构成问题(这也就是我们项目中 Slow 一词)的由来。为了保证模拟硬件行为的正确性,并保证代码的可维护性,我买需要对硬件设备以及其运行过程的各个层次进行抽象,同时使用设计模式使得代码能够被更加灵活地复用和修改。

\subsection{设计模式的意义}
这里以我们项目中的寄存器为例,总结一下设计模式在这个项目中的关键作用:
\begin{enumerate}
    \item 保障行为的准确实现
    \item 方便测试和验证
    \item 减少变更和重构的工作量
\end{enumerate}

\paragraph{保障行为的准确实现} 虽然 6502 只有 6 个寄存器(FLAG,SP,PC,A,X,Y),但这 6 个寄存器在行为上各有特点,但又有着千丝万缕的联系。传统的硬件模拟器中,通常使用结构体直接保存寄存器的内容,然后通过一系列冗长复杂的控制流(例如超过 3 层的 if-else 语句,大块嵌套的 switch-case)等,来实现这些相似但又不完全相同的功能。当然,如果有足够的时间和精力,这些控制流是可以被正确实现的。但在当今的开发节奏下(例如在一周之内完成一轮敏捷开发),这样编写逻辑的方式是行不通的。

因而,我们需要使用设计模式引入合适的抽象,从而将“揉成一团”的逻辑拆分成小块,然后逐一进行实现。我们的代码中,除了极少部分的特殊情况,几乎没有出现超过 100 行的代码块,或者超过 3 层的 if 语句。合理的顶层设计配合语言的特性,加上设计模式的指导,使得准确实现预定的逻辑变得较为容易。

\paragraph{方便测试和验证} 小块封装好的代码除了可以减轻开发人员的心智负担外,也可以方便对于程序的质量控制。一个显而易见的优点是单元测试变得十分简单。测试一个如 \lstinline{getNegativeFlag()} 的简单方法,要远比测试一堆不知所云的 if-else 并试图找出哪里出现了问题要简单的多。

除了单元测试外,集成测试也能从设计模式中受益。在这些小块的方法中,我们可以对逻辑中的不变量(invariant)进行断言,这样在集成测试时,出现的问题可以迅速通过失败的断言被找到。

\paragraph{减少变更和重构的工作量} 一方面,通过一系列合理设计的设计模式构建的源码,本身就为变更和重构提供了丰富的工具。诸如 \lstinline{getNegativeFlag()} 等方法,使得我们重构时可以不必修改大量的实现细节。

另一方面,即便是对于 \lstinline{getNegativeFlag()} 等方法本身的修改,也不必担心对于其他源码的影响。一个我们亲身经历上的例子是在一次大规模的重构后,各个小队依次直接进行了 merge ,没有出现一个冲突,并且成功通过了所有的回归测试。



